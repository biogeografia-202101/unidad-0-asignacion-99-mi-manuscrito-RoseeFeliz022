\documentclass[11pt,]{article}
\usepackage[left=1in,top=1in,right=1in,bottom=1in]{geometry}
\newcommand*{\authorfont}{\fontfamily{phv}\selectfont}
\usepackage[]{mathpazo}


  \usepackage[T1]{fontenc}
  \usepackage[utf8]{inputenc}



\usepackage{abstract}
\renewcommand{\abstractname}{}    % clear the title
\renewcommand{\absnamepos}{empty} % originally center

\renewenvironment{abstract}
 {{%
    \setlength{\leftmargin}{0mm}
    \setlength{\rightmargin}{\leftmargin}%
  }%
  \relax}
 {\endlist}

\makeatletter
\def\@maketitle{%
  \newpage
%  \null
%  \vskip 2em%
%  \begin{center}%
  \let \footnote \thanks
    {\fontsize{18}{20}\selectfont\raggedright  \setlength{\parindent}{0pt} \@title \par}%
}
%\fi
\makeatother




\setcounter{secnumdepth}{3}

\usepackage{longtable,booktabs}

\usepackage{graphicx,grffile}
\makeatletter
\def\maxwidth{\ifdim\Gin@nat@width>\linewidth\linewidth\else\Gin@nat@width\fi}
\def\maxheight{\ifdim\Gin@nat@height>\textheight\textheight\else\Gin@nat@height\fi}
\makeatother
% Scale images if necessary, so that they will not overflow the page
% margins by default, and it is still possible to overwrite the defaults
% using explicit options in \includegraphics[width, height, ...]{}
\setkeys{Gin}{width=\maxwidth,height=\maxheight,keepaspectratio}

\title{Ecología numérica de la familia Myrtaceae en la parcela permanente de
50-ha en Barro Colorado, lago Gatún, Panamá\\
Subtítulo\\
Subtítulo  }



\author{\Large Rosee Aurelina Féliz Méndez\vspace{0.05in} \newline\normalsize\emph{Estudiante, Universidad Autónoma de Santo Domingo (UASD)}  }


\date{}

\usepackage{titlesec}

\titleformat*{\section}{\normalsize\bfseries}
\titleformat*{\subsection}{\normalsize\itshape}
\titleformat*{\subsubsection}{\normalsize\itshape}
\titleformat*{\paragraph}{\normalsize\itshape}
\titleformat*{\subparagraph}{\normalsize\itshape}

\titlespacing{\section}
{0pt}{36pt}{0pt}
\titlespacing{\subsection}
{0pt}{36pt}{0pt}
\titlespacing{\subsubsection}
{0pt}{36pt}{0pt}





\newtheorem{hypothesis}{Hypothesis}
\usepackage{setspace}

\makeatletter
\@ifpackageloaded{hyperref}{}{%
\ifxetex
  \PassOptionsToPackage{hyphens}{url}\usepackage[setpagesize=false, % page size defined by xetex
              unicode=false, % unicode breaks when used with xetex
              xetex]{hyperref}
\else
  \PassOptionsToPackage{hyphens}{url}\usepackage[unicode=true]{hyperref}
\fi
}

\@ifpackageloaded{color}{
    \PassOptionsToPackage{usenames,dvipsnames}{color}
}{%
    \usepackage[usenames,dvipsnames]{color}
}
\makeatother
\hypersetup{breaklinks=true,
            bookmarks=true,
            pdfauthor={Rosee Aurelina Féliz Méndez (Estudiante, Universidad Autónoma de Santo Domingo (UASD))},
             pdfkeywords = {\emph{Myrtaceae}, Ecología numérica},  
            pdftitle={Ecología numérica de la familia Myrtaceae en la parcela permanente de
50-ha en Barro Colorado, lago Gatún, Panamá\\
Subtítulo\\
Subtítulo},
            colorlinks=true,
            citecolor=blue,
            urlcolor=blue,
            linkcolor=magenta,
            pdfborder={0 0 0}}
\urlstyle{same}  % don't use monospace font for urls

% set default figure placement to htbp
\makeatletter
\def\fps@figure{htbp}
\makeatother

\usepackage{pdflscape} \newcommand{\blandscape}{\begin{landscape}}
\newcommand{\elandscape}{\end{landscape}}


% add tightlist ----------
\providecommand{\tightlist}{%
\setlength{\itemsep}{0pt}\setlength{\parskip}{0pt}}

\begin{document}
	
% \pagenumbering{arabic}% resets `page` counter to 1 
%
% \maketitle

{% \usefont{T1}{pnc}{m}{n}
\setlength{\parindent}{0pt}
\thispagestyle{plain}
{\fontsize{18}{20}\selectfont\raggedright 
\maketitle  % title \par  

}

{
   \vskip 13.5pt\relax \normalsize\fontsize{11}{12} 
\textbf{\authorfont Rosee Aurelina Féliz Méndez} \hskip 15pt \emph{\small Estudiante, Universidad Autónoma de Santo Domingo (UASD)}   

}

}








\begin{abstract}

    \hbox{\vrule height .2pt width 39.14pc}

    \vskip 8.5pt % \small 

\noindent Resumen del manuscrito


\vskip 8.5pt \noindent \emph{Keywords}: \emph{Myrtaceae}, Ecología numérica \par

    \hbox{\vrule height .2pt width 39.14pc}



\end{abstract}


\vskip 6.5pt


\noindent  \section{Introducción}\label{introducciuxf3n}

La isla de Barro Colorado (BCI, por sus siglas en inglés) se formó al
término del canal de Panamá en 1974, desde su creación se ha utilizado
como centro de investigación debido a su gran reserva natural. Se
considera monumento natural protegido por el gobierno de Panamá junto a
las penínsulas Peña Blanca, Bohío, Buena Vista, Frijoles y Gigante. La
parcela permanente de 50 hectáreas se encuentra en el bosque húmedo
tropical de la isla de Barro Colorado. Se estableció en 1980, desde
entonces se han realizado 8 censos (aprox. 1 cada 5 años) en cual se
toman en cuenta árboles de tallos leñosos con un diámetro a la altura
del pecho (DAP) mayor a 10 mm, y como resultado en cada censo, se han
identificado, censado y mapeado más de 350, 000 árboles
individuales(Hubbell, Condit, \& Foster, 2021).

Se ha seleccionado el censo número 8 de esta reserva natural por ser el
más reciente y a esta reserva natural en particular debido a la gran
cantidad disponible de datos censales que a través de la Ecología
numérica nos permitirán conocer rasgos básicos de la estructura y
composición de la comunidad de plantas mirtáceas en relación con
factores ambientales.

Las mirtáceas ( \emph{Myrtaceae} Juss) son una familia de plantas
leñosas del orden Myrtales. La mayoría de las especies son árboles,
también hay muchas que son arbustos o subarbustos. Algunas especies
producen flores y frutos, otras raíces adventicias. Se distribuyen
principalmente en zonas tropicales y templadas, con poca representación
en la región africana. La familia cuenta con unos 142 géneros y más de
5.500 especies, incluyendo \emph{Psiloxylon} y \emph{Heteropyxis},
también pueden ser citadas por otros autores como familias monogenéricas
Psiloxylaceae y Heteropyxidaceae. Cabe destacar que la familia integra
los árboles más altos (110-140 m) del planeta ( \emph{Eucalyptus}) y al
género más numeroso (1200‒1800 especies) que existe ( \emph{Syzygium}),
los subarbustos rizomatosos de los géneros de la sabana (
\emph{Psidium}, \emph{Campomanesia} y \emph{Eugenia}), el género
\emph{Metrosideros} que contiene especies arbóreas con muchas raíces
adventicias, y otros géneros son lianas trepadoras de raíces. También
hay un mangle, el monotípico \emph{Osbornia}, un pequeño árbol que
carece de neumatóforos (Wilson, 2010).

R

En este trabajo se harán estudios de asociación, agrupamiento,
diversidad y ecología espacial en relación a factores ambientales con
los datos disponibles del censo número de 8 de la parcela permanente de
50-ha con ecología numérica en R para comprender mejor la estructura y
composición de la comunidad de mirtáceas en la foresta tropical de Barro
Colorado.

\section{Metodología}\label{metodologuxeda}

Ambito geográfico

La parcela permanente de 50 hectáreas es un bosque húmedo tropical que
fue establecido en 1980 por Stephen Hubbell y Robin Foster en la meseta
central de la isla de Barro Colorado (latitud 9\(^\circ\)~9'N, longitud
79\(^\circ\)~50'O). Posee 1,000 m de largo por 500 m de ancho, se divide
en 1250 cuadrantes de 20x20 m (ver figura \ref{fig:mapa_cuadros_bci}).
En la parcela, todos los tallos leñosos con un DAP mayor o igual a 1 cm
se encuentran marcados, enumerados, mapeados e identicados hasta el
nivel de especie. Cada 5 años, esta parcela es censada para evaluar el
cremiento, la mortalidad y el reclutamiento de nuevas generaciones de
plantas. Como resultado de estos censos se han registrado mas de 300
especies de árboles, arbustos y palmas con el próposito de conocer la
historia de vida de las especies, interacciones y dinámica de la
comunidad (Pérez et al., 2005).

\begin{figure}
\centering
\includegraphics[width=0.50000\textwidth]{mapa_cuadros.png}
\caption{Parcela permanente de 50-ha dela isla Barro Colorado, lago
Gatún, Panamá \label{fig:mapa_cuadros_bci}}
\end{figure}

Materiales y Métodos

El análisis exploratorio de datos (AED) permite obtener una visión
general de los datos, transformar o recodificar algunas variables y
orientar los análisis posteriores. Explorar los parámetros simples y las
distribuciones de las variables, ayudará a seleccionar correctamente los
análisis más avanzados(Borcard, n.d.).

Se exploraron los datos del censo número 8 disponibles en la página web
del censo (Hubbell et al., 2021), organizados en dos matrices: la matriz
de comunidad, la cual recopila la información referente a las especies
de la familia \emph{Myrtaceae}, y la matriz ambiental, que contiene la
información referente a las variables de suelo, geomorfologicas,
litologicas y de tipo de habitat. Los analisis se realizaron con ayuda
de los paquetes de R (R Core Team, 2019) para analisis estadísticos y
ecológicos, cabe destacar los paquetes \texttt{vegan} (Oksanen et al.,
2019), \texttt{tidyverse} (Wickham, 2017), \texttt{sf} (Pebesma, 2018),
\texttt{mapview} (Appelhans, Detsch, Reudenbach, \& Woellauer, 2019),
\texttt{leaflet} (Cheng, Karambelkar, \& Xie, 2018), para obtener la
riqueza numérica y abundacia de especies por quadrat y de la comunidad
global, y se correlacionaron con las variables
ambien\textasciitilde{}tales para obtener una visión general de cómo se
distribuyen las mirtáceas en la parcela permanente de BCI.

A. Medición de asociación

Modo Q para datos cuantitativos de especies (abundancia): Matriz de
distancia euclídea, utilizando la transformación * Hellinger *

Modo Q para datos binarios (presencia / ausencia): La índice de
disimilaridad de Jaccard o distancia de Jaccard (** D J **) se puede
expresar como ``la proporción de especies no compartidas''. Como la
distancia de Jaccard ( D J ) es el complemento a 1 de la similaridad de
Jaccard ( S J ), es decir, D J = 1-S J , y dado que arriba calculamos la
distancia, para obtener la similaridad, sólo hay que restarle el valor
de distancia a 1 ( S J = 1-D J ). La fórmula de la similaridad de
Jaccard es S J = a / (a + b + c) , donde a es el número de especies
compartidas (presentes en ambos sitios comparados), b el número de
especies exclusivas del sitio 2, y c el número de especies exclusivas
del sitio 1. Mapa de calor

Modo Q para datos cuantitativos, NO de abundancia de especies (variables
ambientales) variables de suelo, todas cuantitativas, puntuaciones z

Modo Q para datos cualitativos y cuantitativos (mixtos), NO de
abundancia de especies (variables ambientales)
hetereogeneidad\_ambiental, habitat y quebrada.

Modo R para datos cuantitativos de especies (abundancia) grado de
asociación entre especies, NO entre sitios transformación Chi a la
matriz de comunidad transpuesta, se calcula la distancia euclidea y se
crea el mapa de calor.

Modo R para datos cuantitativos, NO de abundancia de especies (variables
ambientales) índice de correlación de Pearson. Sin embargo, si los datos
no presentan distribución normal, puedes emplear métricas más flexibles,
como el índice rho de Spearman

B. Agrupamiento (cluster analysis): Correlación cofenética Anchura de
silueta UPGMA WARD Remuestro de bootstrap y BP Remuestreo pro medio de
Bootstrap multiescalar y AU Para UPGMA: Homogeneidad de promedios,
pruebas t (medias), distribucion t de student, suma de rangos de
wilcoxon (medianas) Para WARD: ANOVA y Kruskal-Wallis Analisis de
especies indicadoras INDVal Especies con preferencia de habitat:
coeficiente de correlacion biserial puntual

C. Diversidad: Diversidad alpha Matriz de correlacion de Pearson Modelos
de abundancia de especie Rarefaccion- Curva de rarefaccion Riqueza de
especies estimación y comparación, completitud de muestra Enfoques
ansioticos- Chao Enfoques no ansioticos- Chao

E. Ecología espacial: Autocorrelacion espacial mediante correlograma
Autocorrelacion mediante Prueba de Mantel (Matrices de distancia) I de
Moran global aplicado a abundancia de especies transformadas sin
tendencia Yo de Moran local

\section{Resultados}\label{resultados}

Preguntas de investigación A. Medición de asociación:

¿Se detectan especies asociadas dentro de mi familia seleccionada?
¿Existe asociación entre variables ambientales/atributos? ¿Cuáles
variables?

B. Agrupamiento (cluster analysis): Los cuadros (o quadrats) de 1
hectárea, ¿se organizan en grupos discontinuos según la composición de
las especies de mi familia seleccionada? Si existe algún patrón, ¿es
consistente con alguna variable ambiental/atributo? ¿Hay especies
indicadoras o con preferencia por determinadas condiciones
ambientales/atributos?

C. Diversidad: Según los análisis de estimación de riqueza, ¿está
suficientemente representada mi familia? Consideremos como buena
representación un 85\% ¿Existe asociación de la diversidad alpha con
variables ambientales/atributos? ¿Con cuáles? ¿Existe contribución local
o por alguna especie a la diversidad beta?

E. Ecología espacial: ¿Alguna(s) especies de mi familia presenta(n)
patrón aglomerado? ¿Cuál(es)? ¿Se asocia con alguna variable? ¿Predicen
bien la ocurrencia de dicha(s) especie(s) los modelos de distribución de
especies (SDM)?

La tabla \ref{tab:abun_sp} y la figura \ref{fig:abun_sp_q} visualiza
estos resultados a continuación.

\begin{longtable}[]{@{}lr@{}}
\caption{\label{tab:abun_sp}Abundancia por especie de la familia
Myrtaceae}\tabularnewline
\toprule
Latin & n\tabularnewline
\midrule
\endfirsthead
\toprule
Latin & n\tabularnewline
\midrule
\endhead
Eugenia galalonensis & 1975\tabularnewline
Eugenia oerstediana & 1838\tabularnewline
Eugenia coloradoensis & 609\tabularnewline
Chamguava schippii & 541\tabularnewline
Eugenia nesiotica & 502\tabularnewline
Psidium friedrichsthalianum & 58\tabularnewline
Myrcia gatunensis & 56\tabularnewline
\bottomrule
\end{longtable}

\begin{figure}
\centering
\includegraphics{manuscrito_files/figure-latex/unnamed-chunk-3-1.pdf}
\caption{\label{fig:abun_sp_q}Abundancia de especies por quadrat}
\end{figure}

Patrón de riqueza\ldots{} (ver Figura
\ref{fig:mapa_cuadros_riq_mi_familia})

\begin{figure}
\centering
\includegraphics[width=0.50000\textwidth]{mapa_cuadros_riq_mi_familia.png}
\caption{Distribución de la riqueza de la familia \emph{Myrtaceae}
\label{fig:mapa_cuadros_riq_mi_familia}}
\end{figure}

Rangos de ph\ldots{} (ver Figuras \ref{fig:mapa_cuadros_ph} y
\ref{fig:mapa_cuadros_pendiente}).

\begin{figure}
\centering
\includegraphics[width=0.50000\textwidth]{mapa_cuadros_ph.png}
\caption{Distribución del pH por cuadros de 1 Ha
\label{fig:mapa_cuadros_ph}}
\end{figure}

\begin{figure}
\centering
\includegraphics[width=0.50000\textwidth]{mapa_cuadros_pendiente.png}
\caption{Distribución de las pendientes (en grados) por cuadro de 1 Ha
\label{fig:mapa_cuadros_pendiente}}
\end{figure}

En la matriz, \ldots{} (ver figura
\ref{fig:matriz_disimilaridad_jacard})

\begin{figure}
\centering
\includegraphics{matriz_disimilaridad_jacard.png}
\caption{Matriz de Disimilaridad de Jacard
\label{fig:matriz_disimilaridad_jacard}}
\end{figure}

\begin{figure}
\centering
\includegraphics{sitios_de_BCI_segun_composicion_de_especies_de_Myrtaceae_metodo_de_Ward_a_partir_de_matriz_de_distancia_de_cuerdas.png}
\caption{Sitios de BCI según composición de especies de Myrtaceae Método
de Ward a partir de matriz de distancia de cuerdas}
\end{figure}

\begin{figure}
\centering
\includegraphics{mapa_ph.png}
\caption{Mapa de las concentraciones de pH}
\end{figure}

{[}{]} \# Discusión

\section{Agradecimientos}\label{agradecimientos}

\section{Información de soporte}\label{informaciuxf3n-de-soporte}

\ldots

\section{\texorpdfstring{\emph{Script}
reproducible}{Script reproducible}}\label{script-reproducible}

\ldots

\section*{Referencias}\label{referencias}
\addcontentsline{toc}{section}{Referencias}

\hypertarget{refs}{}
\hypertarget{ref-mapview}{}
Appelhans, T., Detsch, F., Reudenbach, C., \& Woellauer, S. (2019).
\emph{Mapview: Interactive viewing of spatial data in r}. Retrieved from
\url{https://CRAN.R-project.org/package=mapview}

\hypertarget{ref-borcard2018numerical}{}
Borcard, F. ~. ois y L., Daniel y Gillet. (n.d.). \emph{Ecología
numérica con r} (Springer, Ed.).

\hypertarget{ref-leaflet}{}
Cheng, J., Karambelkar, B., \& Xie, Y. (2018). \emph{Leaflet: Create
interactive web maps with the javascript 'leaflet' library}. Retrieved
from \url{https://CRAN.R-project.org/package=leaflet}

\hypertarget{ref-webcenso}{}
Hubbell, S., Condit, R., \& Foster, R. (2021). Forest Census Plot on
Barro Colorado Island. Retrieved May 5, 2021, from
\url{http://ctfs.si.edu/webatlas/datasets/bci/}

\hypertarget{ref-vegan}{}
Oksanen, J., Blanchet, F. G., Friendly, M., Kindt, R., Legendre, P.,
McGlinn, D., \ldots{} Wagner, H. (2019). \emph{Vegan: Community ecology
package}. Retrieved from \url{https://CRAN.R-project.org/package=vegan}

\hypertarget{ref-sf}{}
Pebesma, E. (2018). Simple Features for R: Standardized Support for
Spatial Vector Data. \emph{The R Journal}, \emph{10}(1), 439--446.
\url{https://doi.org/10.32614/RJ-2018-009}

\hypertarget{ref-perez2005metodologia}{}
Pérez, R., Aguilar, S., Condit, R., Foster, R., Hubbell, S., \& Lao, S.
(2005). Metodologia empleada en los censos de la parcela de 50 hectareas
de la isla de barro colorado, panamá. \emph{Centro de Ciencias
Forestales Del Tropico (CTFS) Y Instituto Smithsonian de Investigaciones
Tropicales (STRI)}, 1--24.

\hypertarget{ref-citadeR}{}
R Core Team. (2019). \emph{R: A language and environment for statistical
computing}. Retrieved from \url{https://www.R-project.org/}

\hypertarget{ref-tidyverse}{}
Wickham, H. (2017). \emph{Tidyverse: Easily install and load the
'tidyverse'}. Retrieved from
\url{https://CRAN.R-project.org/package=tidyverse}

\hypertarget{ref-wilson2010myrtaceae}{}
Wilson, P. G. (2010). Myrtaceae. In \emph{Flowering plants. eudicots}
(pp. 212--271). Springer.




\newpage
\singlespacing 
\end{document}
