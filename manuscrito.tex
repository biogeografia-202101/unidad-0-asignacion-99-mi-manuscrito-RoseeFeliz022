\documentclass[11pt,]{article}
\usepackage[left=1in,top=1in,right=1in,bottom=1in]{geometry}
\newcommand*{\authorfont}{\fontfamily{phv}\selectfont}
\usepackage[]{mathpazo}


  \usepackage[T1]{fontenc}
  \usepackage[utf8]{inputenc}



\usepackage{abstract}
\renewcommand{\abstractname}{}    % clear the title
\renewcommand{\absnamepos}{empty} % originally center

\renewenvironment{abstract}
 {{%
    \setlength{\leftmargin}{0mm}
    \setlength{\rightmargin}{\leftmargin}%
  }%
  \relax}
 {\endlist}

\makeatletter
\def\@maketitle{%
  \newpage
%  \null
%  \vskip 2em%
%  \begin{center}%
  \let \footnote \thanks
    {\fontsize{18}{20}\selectfont\raggedright  \setlength{\parindent}{0pt} \@title \par}%
}
%\fi
\makeatother




\setcounter{secnumdepth}{3}

\usepackage{longtable,booktabs}

\usepackage{graphicx,grffile}
\makeatletter
\def\maxwidth{\ifdim\Gin@nat@width>\linewidth\linewidth\else\Gin@nat@width\fi}
\def\maxheight{\ifdim\Gin@nat@height>\textheight\textheight\else\Gin@nat@height\fi}
\makeatother
% Scale images if necessary, so that they will not overflow the page
% margins by default, and it is still possible to overwrite the defaults
% using explicit options in \includegraphics[width, height, ...]{}
\setkeys{Gin}{width=\maxwidth,height=\maxheight,keepaspectratio}

\title{Ecología numérica de la familia Myrtaceae en la parcela permanente de
50-ha en Barro Colorado, lago Gatún, Panamá\\
Subtítulo\\
Subtítulo  }



\author{\Large Rosee Aurelina Féliz Méndez\vspace{0.05in} \newline\normalsize\emph{Estudiante, Universidad Autónoma de Santo Domingo (UASD)}  }


\date{}

\usepackage{titlesec}

\titleformat*{\section}{\normalsize\bfseries}
\titleformat*{\subsection}{\normalsize\itshape}
\titleformat*{\subsubsection}{\normalsize\itshape}
\titleformat*{\paragraph}{\normalsize\itshape}
\titleformat*{\subparagraph}{\normalsize\itshape}

\titlespacing{\section}
{0pt}{36pt}{0pt}
\titlespacing{\subsection}
{0pt}{36pt}{0pt}
\titlespacing{\subsubsection}
{0pt}{36pt}{0pt}





\newtheorem{hypothesis}{Hypothesis}
\usepackage{setspace}

\makeatletter
\@ifpackageloaded{hyperref}{}{%
\ifxetex
  \PassOptionsToPackage{hyphens}{url}\usepackage[setpagesize=false, % page size defined by xetex
              unicode=false, % unicode breaks when used with xetex
              xetex]{hyperref}
\else
  \PassOptionsToPackage{hyphens}{url}\usepackage[unicode=true]{hyperref}
\fi
}

\@ifpackageloaded{color}{
    \PassOptionsToPackage{usenames,dvipsnames}{color}
}{%
    \usepackage[usenames,dvipsnames]{color}
}
\makeatother
\hypersetup{breaklinks=true,
            bookmarks=true,
            pdfauthor={Rosee Aurelina Féliz Méndez (Estudiante, Universidad Autónoma de Santo Domingo (UASD))},
             pdfkeywords = {Myrtaceae, Ecología numérica, mirtáceas, parcela permanente de 50-ha,
BCI},  
            pdftitle={Ecología numérica de la familia Myrtaceae en la parcela permanente de
50-ha en Barro Colorado, lago Gatún, Panamá\\
Subtítulo\\
Subtítulo},
            colorlinks=true,
            citecolor=blue,
            urlcolor=blue,
            linkcolor=magenta,
            pdfborder={0 0 0}}
\urlstyle{same}  % don't use monospace font for urls

% set default figure placement to htbp
\makeatletter
\def\fps@figure{htbp}
\makeatother

\usepackage{pdflscape} \newcommand{\blandscape}{\begin{landscape}}
\newcommand{\elandscape}{\end{landscape}}


% add tightlist ----------
\providecommand{\tightlist}{%
\setlength{\itemsep}{0pt}\setlength{\parskip}{0pt}}

\begin{document}
	
% \pagenumbering{arabic}% resets `page` counter to 1 
%
% \maketitle

{% \usefont{T1}{pnc}{m}{n}
\setlength{\parindent}{0pt}
\thispagestyle{plain}
{\fontsize{18}{20}\selectfont\raggedright 
\maketitle  % title \par  

}

{
   \vskip 13.5pt\relax \normalsize\fontsize{11}{12} 
\textbf{\authorfont Rosee Aurelina Féliz Méndez} \hskip 15pt \emph{\small Estudiante, Universidad Autónoma de Santo Domingo (UASD)}   

}

}








\begin{abstract}

    \hbox{\vrule height .2pt width 39.14pc}

    \vskip 8.5pt % \small 

\noindent El objetivo general es conocer los rasgos básicos de la estructura y
composición de la comunidad de mirtáceas en relación con factores
ambientales de la parcela permanente de 50-ha de isla Barro Colorado. Se
hiceron estudios para medir el grado de asociación, agrupamiento,
diversidad y ecología espacial de esta familia con la ayuda de los
paquetes de R y con los datos de octavo censo de esta localidad. Las
mirtáceas presentaron una riqueza de 7 especies con una abundancia de
5579 individuos. Las especies del género \emph{Eugenia} presentaron
altos grados de asociación. El agrupamiento Ward de varianza mínima
sugirió la partición de 4 grupos que alcazaron el 100\% de la
completitud de muestra. La diversidad de mirtáceas posee una correlación
positiva con \emph{Al, P, Ca, Fe} y la geomorfología de pendiente media.
Las especies \emph{Changuava schipii} y \emph{E. oerstediana} son las
especies que aportan a la diversidad alpha y están estrechamente
relacionadas con los sitios que aportan a la misma. Las mirtáceas
presentan patrones aglomerados para vecinos de primer orden, a excepción
de \emph{M. gatunencis}, que presentó un patrón aleatorio. El modelo de
abundancia de especies muestra que el 56\% de la comunidad presenta
mayores valores de equidad (log normal 10\% y null 46\%).


\vskip 8.5pt \noindent \emph{Keywords}: Myrtaceae, Ecología numérica, mirtáceas, parcela permanente de 50-ha,
BCI \par

    \hbox{\vrule height .2pt width 39.14pc}



\end{abstract}


\vskip 6.5pt


\noindent  \section{Introducción}\label{introducciuxf3n}

La isla de Barro Colorado (BCI, por sus siglas en inglés) se formó al
término del canal de Panamá en 1914, desde su creación se ha utilizado
como centro de investigación debido a su gran reserva natural. Se
considera monumento natural protegido por el gobierno de Panamá junto a
las penínsulas Peña Blanca, Bohío, Buena Vista, Frijoles y Gigante
(Smithsonian Tropical Research Institute, 2010). La parcela permanente
de 50 hectáreas se encuentra en el bosque húmedo tropical de la isla de
Barro Colorado. Se estableció en 1980, desde entonces se han realizado 8
censos (aprox. 1 cada 5 años), en los cuales se toman en cuenta árboles
de tallos leñosos con un diámetro a la altura del pecho (DAP) mayor a 10
mm, y como resultado en cada censo, se han identificado, censado y
mapeado más de 350,000 árboles individuales (Hubbell, Condit, \& Foster,
2021).

Las mirtáceas (Myrtaceae Juss) son una familia de plantas leñosas del
orden Myrtales, presentes en la parcela permanente de BCI. La mayoría de
las especies son árboles, también hay muchas que son arbustos o
subarbustos. Algunas especies producen flores y frutos, otras raíces
adventicias. Se distribuyen principalmente en zonas tropicales y
templadas, con poca representación en la región africana. La familia
cuenta con unos 142 géneros y más de 5.500 especies, incluyendo
\emph{Psiloxylon} y \emph{Heteropyxis}, también pueden ser citadas por
otros autores como familias monogenéricas Psiloxylaceae y
Heteropyxidaceae. Cabe destacar que la familia integra los árboles más
altos (110-140 m) del planeta (\emph{Eucalyptus}) y al género más
númeroso (1200‒1800 especies) que existe (\emph{Syzygium}), los
subarbustos rizomatosos de los géneros de la sabana (\emph{Psidium},
\emph{Campomanesia} y \emph{Eugenia}), el género \emph{Metrosideros} que
contiene especies arbóreas con muchas raíces adventicias, y otros
géneros son lianas trepadoras de raíces. También hay un mangle, el
monotípico \emph{Osbornia}, un pequeño árbol que carece de neumatóforos
(Wilson, 2010).

En BCI, durante los repetidos censos que se han realizado en la parcela
permanente de 50-ha desde 1981b (Hubbell et al., 2021), se han reportado
varias especies de mirtáceas, siendo su representatividad relativamente
importante en dicho enclave.

Mediante análisis exploratorios preliminares, se detectaron patrones
singulares de distribución espacial y de diversidad. Por ejemplo,
algunas especies (e.g. \emph{Changuava schipii}), mostraron una
distribución espacial concentrada. Igualmente, varias especies mostraron
preliminarmente asociación con variables de suelo y geomorfólógicas.
Ninguna investigación ha reportado ni analizado estos patrones a la
fecha. El objetivo de este estudio es caracterizar la comunidad de
mirtáceas de BCI, con ayuda de técnicas de ecología numérica
(e.g.~análisis de asociación, agrupamiento, diversidad y ecología
espacial en relación a factores ambientales), utilizando como fuente los
datos disponibles del censo número de 8 de la parcela permanente de
50-ha.

\section{Metodología}\label{metodologuxeda}

\subsection{Ámbito geográfico}\label{uxe1mbito-geogruxe1fico}

Los datos analizados fueron colectados en la parcela permanente de 50
hectáreas de la isla Barro Colorado (BCI en lo adelante). La parcela
contiene 50 ha de bosque húmedo tropical, situado en la meseta central
de la isla, y fue establecida en 1980 por Stephen Hubbell y Robin Foster
en la meseta central de la isla de Barro Colorado (latitud
9\(^\circ\)~9'N, longitud 79\(^\circ\)~51'O). Posee 1,000 m de largo por
500 m de ancho, se divide en 1250 cuadrantes de 20x20 m (ver figura
\ref{fig:mapa_cuadros_bci}). En la parcela, todos los tallos leñosos con
un diámetro a la altura de pecho (DAP) mayor o igual a 1 cm se
encuentran marcados, enumerados, mapeados e identicados hasta el nivel
de especie. Cada 5 años, esta parcela es censada para evaluar el
crecimiento, la mortalidad y para el reclutamiento de nuevas
generaciones de plantas. Como resultado de estos censos se han
registrado mas de 300 especies de árboles, arbustos y palmas con el
próposito de conocer la historia de vida de las especies, interacciones
y dinámica de la comunidad (Pérez et al., 2005).

\begin{figure}
\centering
\includegraphics[width=0.50000\textwidth]{mapa_cuadros.png}
\caption{Parcela permanente de 50-ha dela isla Barro Colorado, lago
Gatún, Panamá \label{fig:mapa_cuadros_bci}}
\end{figure}

\subsection{Materiales y Métodos}\label{materiales-y-muxe9todos}

Se ha seleccionado el censo número 8 de esta reserva natural por ser el
más reciente y a esta reserva natural en particular debido a la gran
cantidad disponible de datos censales que a través de la Ecología
numérica nos permitirán conocer rasgos básicos de la estructura y
composición de la comunidad de plantas mirtáceas en relación con
factores ambientales.

Se exploraron los datos del censo número 8 disponibles en la página web
del censo (Hubbell et al., 2021), organizados en dos matrices: la matriz
de comunidad, la cual recopila la información referente a las especies
de la parcela permanente de 50-ha, y la matriz ambiental, que contiene
la información referente a las variables de suelo, geomorfológicas,
litológicas y de tipo de habitat. Los análisis, tablas, figuras y
gráficos se realizaron con los scripts de análisis de José R. Martínez
(Batlle, 2020) y con ayuda de los paquetes de R para análisis
estadísticos y ecológicos (R Core Team, 2019), cabe destacar los
paquetes \texttt{vegan} (Oksanen et al., 2019), \texttt{tidyverse}
(Wickham, 2017), \texttt{sf} (Pebesma, 2018), \texttt{mapview}
(Appelhans, Detsch, Reudenbach, \& Woellauer, 2019) y \texttt{leaflet}
(Cheng, Karambelkar, \& Xie, 2018) que fueron los más utilizados.

En los análisis de medición de asociación en modo Q, se utilizaron
varias distancias, como ji-cuadrado, normalizada, Hellinger y Jaccard.
Las tres primeras son distancias euclideas, calculadas sobre los datos
transformados, apropiadas tanto para los datos cuantitativos como para
los datos de presencia-ausencia; y la última, la distancia de Jaccard
(\(D_J\)) se puede expresar como la proporción de especies no
compartidas. La distancia de Jaccard es el complemento a 1 de la
similaridad de Jaccard (\(S_J\)), es decir, \(D_J\) = 1- \(S_J\) , de
esta manera para obtener la similaridad, sólo hay que restarle el valor
de distancia a 1 (\(S_J\) = 1- \(D_J\)). Se puede usar para evaluar la
distancia entre especies, usando como fuente la matriz de comunidad
transpuesta convertida a binaria (presencia / ausencia) (Borcard, 2018).

Para el análisis de medición de asociación en modo R se utilizó el
coeficiente de correlación de Pearson, el cual tiene como objetivo medir
la fuerza o grado de asociación entre dos variables aleatorias
cuantitativas que poseen una distribución normal bivariada conjunta.
Alternativamente cuando este no cumple con los supuestos se utiliza
coeficiente de correlación no paramétrico de Spearman, se define como el
coeficiente de correlación lineal entre los rangos Ri(x) y Ri(y)
(Restrepo \& González, 2007).

Se realizaron análisis de agrupamiento utilizando distintos métodos
(e.g.~UPGMA, Ward) para explorar la estructura de la comunidad en
función de su composición. Para elegir entre métodos se utilizó la
correlación cofenética; se consideró al agrupamiento con la correlación
más alta como aquel que retiene la mayor parte de la información
contenida en la matriz de disimilitud; no obstante, esto no significa
necesariamente que este método sea el más adecuado para el objetivo del
investigador. Luego para escoger una cantidad óptima de clusters para
cada agrupamiento se utilizó la anchura de la silueta, ésta es una
medida del grado de pertenencia de un objeto a su clúster, basada en la
disimilitud media entre este objeto y el clúster al que pertenece,
comparada con la misma medida del clúster más próximo (Borcard, 2018).

Los métodos aglomerativos utilizados para constatar y evaluar los grupos
que hacían sentido para las mirtáceas de este estudio son desarrollados
a continuación:

-El método aglomerativo por enlace simple (\emph{single}), conocido como
la clasificación por vecinos más cercanos, aglomera objetos en función
de sus disimilitudes más cortas entre pares: la fusión de un objeto con
un grupo en un nivel de disimilitud determinado sólo requiere que un
objeto de cada grupo que se aglomerare esté vinculado al otro en ese
nivel. En consecuencia, el dendrograma resultante de una aglomeración de
enlace simple suele mostrar encadenamiento de objetos. La lista de las
primeras conexiones que hacen a un objeto miembro de un clúster, o que
permite la fusión de dos clústeres, se denomina cadena de conexiones
primarias; esta cadena forma el árbol de expansión mínima (MST).

-El método aglomerativo por enlace completo (\emph{complete}), conocido
como la clasificación del vecino más lejano, permite que un objeto se
agrupe con otro grupo sólo en la disimilitud correspondiente a la del
par de objetos más distante; de esta manera con mayor motivo, todos los
miembros de los dos grupos están vinculados. Un grupo admite un nuevo
miembro sólo a una disimilitud correspondiente al objeto más lejano del
grupo. De ello se deduce que cuánto más grande es un grupo, más difícil
es aglomerarse con él. La vinculación completa resulta en muchos grupos
pequeños separados que se aglomeran a grandes distancias, por lo que
este método es interesante para buscar e identificar discontinuidades en
los datos.

-El método de grupos de pares no ponderados con media aritmética (UPGMA,
por sus siglas en inglés) es el más conocido de la familia métodos
aglomerativos por enlace promedio, éstos se basan en las disimilitudes
medias entre los objetos o en los centroides de los grupos. El método
UPGMA permite que un objeto se una a un grupo en la media de las
disimilitudes entre este objeto y todos los miembros del grupo. Cuando
dos grupos se unen, lo hacen a la media de las disimilitudes entre todos
los miembros de un grupo y todos los miembros del otro.

-El método de agrupación de varianza mínima de Ward se basa en el
criterio del modelo lineal de mínimos cuadrados. Su objetivo es definir
los grupos de tal manera que la suma de cuadrados dentro del grupo (es
decir, el error cuadrático del ANOVA) se minimiza. La suma de errores al
cuadrado dentro del grupo puede calcularse como la suma de las
distancias al cuadrado entre los miembros de un grupo dividido por el
número de objetos. Este método fue seleccionado porque produce grupos
con números de elementos más equilibrados, o que evita los grupos de
pocos elementos (Borcard, 2018).

El remuestreo \emph{bootstrap} consiste en muestrear aleatoriamente
subconjuntos de los datos y calcular la agrupación en estos
subconjuntos. Luego de repetir este proceso un gran número de veces, se
cuenta la proporción de los resultados de clustering replicados en los
que aparece un cluster determinado. Esta proporción se denomina
probabilidad \emph{bootstrap} (BP) del cluster. Adicionalmente, se
aplicó el remuestreo \emph{bootstrap} multiescalar, utiliza muestras
\emph{bootstrap} de varios tamaños diferentes para estimar el valor p de
cada conglomerado. Esta mejora produce valores p ``aproximadamente
insesgados'' (AU) (Borcard, 2018).

Para evaluar homogeneidad de promedios de las variables ambientales
entre los grupos Ward y las variables ambientales fueron ANOVA, que
evalúa homogeneidad de medias, y Kruskal-Wallis, que evalúa la
homogeneidad de medianas; los cuales hacen sentido para agrupamientos de
3 grupos o más (Batlle, 2020).

El análisis de especies indicadoras de los grupos Ward se hizo mediante
el método del Valor Indicador (en lo adelante, IndVal), el cual se
calcula como el producto de la especificidad de una especie para el
grupo objetivo por su fidelidad al grupo objetivo. La especificidad se
define por la abundancia media de la especie dentro del grupo objetivo
comparada con su abundancia media en todos los grupos; la fidelidad es
la proporción de sitios del grupo objetivo en el que está presente la
especie. Y el análisis de especies con preferencia por hábitat se
realizó mediante el coeficiente de correlación biserial puntual
(Borcard, 2018).

Para medir la diversidad alpha se utilizaron los índices de diversidad,
descritos a continuación:

-La equidad de Pielou (denominada también equidad de Shannon) equivale a
\(J=H_1/H_0\).

-Los tres primeros números de diversidad de Hill : \(N_0 =q\) (la
riqueza de especies), \(N_1 = e^H\) (número de especies abundantes), y
\(N_1 = 1/\)\(\lambda\) (inverso de Simpson).

-Los ratios de Hill: \(E_1 = N_1/N_0\) (versión de la equidad de
Shannon) y \(E_2 = N_2/N_0\) (versión de la equidad de Simpson).

Basado en los supuestos de Whittaker, según los cuales la diversidad
beta es la variación espacial de la diversidad entre sitios, la
diversidad beta fue medida en función de las especies y sitios que
contribuían a ésta con la función
\texttt{determinar\_contrib\_local\_y\_especie} de R del script fuente
(Batlle, 2020).

Para medir los patrones espaciales de las especies y las variables
ambientales se utilizaron tanto el correlograma, como la prueba Mantel,
el índice de autocorrelación I de Moran y los mapas de indicadores
locales de autocorrelación espacial (en lo adelante Mapas LISA). Primero
se aplicó la prueba I de Moran que está contenida en la función
\texttt{calcular\_autocorrelacion}. Luego, se aplicó a las variables
ambientales y las abundancias de especies transformadas sin tendencia,
lo que resultó en unos clusters LISA que mostraron los patrones
significativos de los que se pueden inferir las dependencias inducidas.

\section{Resultados}\label{resultados}

La familia Myrtaceae está presente en la parcela permanente de 50-ha de
BCI con una abundancia de 5,579 individuos pertenecientes a 7 especies,
de las cuales las más abundantes fueron \emph{Eugenia galalonensis} y
\emph{Eugenia oerstediana}, representadas con 1,975 y 1,838 individuos
cada una, y las especies más raras fueron \emph{Psidium
friedrichsthalianum} y \emph{Myrcia gatunensis}, con 58 y 56 individuos
respectivamente (ver figura \ref{tab:abun_sp}).

\begin{longtable}[]{@{}lr@{}}
\caption{\label{tab:abun_sp}Abundancia por especie de la familia
Myrtaceae}\tabularnewline
\toprule
Latin & n\tabularnewline
\midrule
\endfirsthead
\toprule
Latin & n\tabularnewline
\midrule
\endhead
Eugenia galalonensis & 1975\tabularnewline
Eugenia oerstediana & 1838\tabularnewline
Eugenia coloradoensis & 609\tabularnewline
Chamguava schippii & 541\tabularnewline
Eugenia nesiotica & 502\tabularnewline
Psidium friedrichsthalianum & 58\tabularnewline
Myrcia gatunensis & 56\tabularnewline
\bottomrule
\end{longtable}

\begin{figure}
\centering
\includegraphics{manuscrito_files/figure-latex/unnamed-chunk-3-1.pdf}
\caption{\label{fig:abun_sp_q}Abundancia de especies por quadrat}
\end{figure}

La distancia de \emph{ji}-cuadradado y la distancia de Jacard resultaron
pequeñas entre especies del genéro \emph{Eugenia} (\emph{E.
oerstediana}, \emph{E. galalonensis}, \emph{E. nesiotica} y \emph{E.
coloradoensis}), lo cual sugiere un patrón de dependencia, debido a que
tienen altos grados de asociación; y las especies \emph{Psidium
friedrichsthalianum}, \emph{Myrcia gatunensis} y \emph{Changuava
schippii} presentan un posible patrón independiente, no parecen
asociarse con otras (ver figura \ref{fig:matriz_Jacard}). La riqueza de
la familia presentó asociación estadística, a través del índice de
Spearman, en términos positivos con \emph{Al}, * P* y en términos
negativos con \emph{Ca}; y la abundancia de mi familia presentó
asociación estadística, a través del índice de Spearman y el índice de
Pearson, en términos positivos con \emph{Al} y elevación media, y en
términos negativos con \emph{Ca}, heterogeneidad ambiental y
geomorfología de vaguada (ver figuras \ref{fig:matriz_spearman} y
\ref{fig:matriz_pearson}).

\begin{figure}
\centering
\includegraphics{Disimilaridad_.png}
\caption{Matriz de disimilaridad de Jacard \label{fig:matriz_Jacard}}
\end{figure}

\begin{figure}
\centering
\includegraphics{matriz_correlacion_suelo_abun_riq_spearman.png}
\caption{Matriz de correlación, índice de Spearman
\label{fig:matriz_spearman}}
\end{figure}

\begin{figure}
\centering
\includegraphics{matriz_correlacion_geomorf_abun_riq_spearman.png}
\caption{Matriz de correlación, índice de Pearson
\label{fig:matriz_pearson}}
\end{figure}

El método de agrupamiento Ward de varianza mínima, conjuntamente con el
mapa de calor (ver figura \ref{fig:mapadecalor_ward}), mostró que las
mirtáceas de la parcela permanente de 50-ha de BCI se distribuyen en 4
grupos, de 2, 13, 15 y 20 sitios, respectivamente (ver figura
\ref{fig:mapa_ward}). Los métodos de agrupamiento aglomerativos por
enlace simple, por enlace completo y por enlace promedio (grupos de
pares no ponderados con media aritmética, UPGMA por sus siglas en
inglés) destacaron la singularidad de este grupo formado por dos sitios
(14 y 19). Además, el muestreo de \emph{bootstrap} multiescalar respalda
este grupo con un probabilidad de \emph{bootstrap} (BP) de 76 \% y
probabilidad de valores aproximadamente insesgados (AU) de 99 \%, de que
sea un grupo real (ver figura \ref {fig:*bootstrap*_multiescalar}). Las
mirtáceas presentaron asociación estadística, según el diagrama de
cajas, con un conjunto de variables de suelo (\emph{Al, Fe, Mn, N.
min.}, etc.) y atributos del terreno (curvatura perfil media, curvatura
tangencial media, elevación media, etc.) (ver figura
\ref{fig:ward_con_variables}). ANOVA, Kruskal-Walis*

\begin{figure}
\centering
\includegraphics[width=0.50000\textwidth]{Mapadecalor_Ward_aa2.png}
\caption{Mapa de calor con el dendrograma del agrupamiento Ward
\label{fig:mapadecalor_ward}}
\end{figure}

\begin{figure}
\centering
\includegraphics[width=0.50000\textwidth]{mapa_ward_k4.png}
\caption{Agrupamiento por el método Ward de varianza mínima de las
mirtáceas \label{fig:mapa_ward}}
\end{figure}

\begin{figure}
\centering
\includegraphics{bootstrap_Ward.png}
\caption{Dendrograma, agrupamiento Ward con los porcentajes del
remuestreo de \emph{bootstrap} multiescalar
\label{fig:*bootstrap*_multiescalar}}
\end{figure}

\begin{figure}
\centering
\includegraphics{correlograma_wardyvariablesambientales.png}
\caption{Diagrama de cajas de los grupos Ward en relación con variables
ambientales y atributos \label{fig:ward_con_variables}}
\end{figure}

Para este agrupamiento, el análisis de especies indicadoras mediante
IndVal para una significancia menor de 0.05, propuso como especie
asociada como indicadora del grupo 3 a \emph{Chamguava schippii}, para
el conjunto de grupos 1+2 \emph{Eugenia coloradoensis} y para el
conjunto de grupos 3+4 \emph{Eugenia oerstediana}; y el análisis de
especies con preferencia por hábitat mediante el coeficiente de
correlación biserial puntual para una significancia menor de 0.005,
sugirió que \emph{Eugenia coloradoensis} tiene preferencia por el grupo
2, \emph{Chamguava schippii} por el grupo 3 y \emph{Eugenia oerstediana}
por el grupo 4.

Según los modelos de estimación de riqueza (\emph{Homogeneous model},
\emph{Homogeneous (MLE)}, los Chao y los Jacknife), la completitud de
muestra se alcanzó al 100\% para las mirtáceas de este ámbito geográfico
por lo que no sería necesario aumentar el esfuerzo de muestreo ya que no
se espera encontrar otras especies en BCI*. La diversidad alpha para el
agrupamiento Ward, los cuatro grupos presentaban la riqueza máxima (7
especies) con diferentes abundancias (1882, 1205, 553 y 1939,
respectivamente). Para los grupos Ward, la riqueza máxima fue estimada y
observada, por lo que también se alcanzó la completitud de muestra al
100\% y al 98\% para el grupo 3 (grupo con la menor abundancia), y no
será necesario aumentar los esfuerzos de muestreo.

La riqueza (\(N_0\)), \(E_2\) y \(N_2\) de Hill sugieren que la
diversidad de mirtáceas presenta una correlación positiva importante con
\emph{Al, P, Ca y Fe}, en suma la equidad de Pielou (J), los ratios de
Hill (\(E_1\) y \(E_2\)) y \(N_2\) infieren una correlación positiva
notable con la presencia de la geomorfología de pendiente media.

\emph{Changuava schipii} y \emph{Eugenia oerstediana} son las especies
que hacen contribución a la diversidad beta, éstas están bien
representadas (la primera con gran dominancia) en los sitios 14 y 19
(grupo 3 Ward) que hacen contribución a la diversidad beta; el 14 fue
uno de los cinco sitios que poseen la riqueza máxima (los demás sitios
son 13, 17, 22 y 40) y el 19 fue el sitio más abundante con 399
individuos, de los cuales 261 pertenecían a \emph{C. schipii} (ver
figura \ref{fig:abun_sp_q}). Además, los sitios 14 y 19 están ubicados
uno al lado del otro. El modelo de abundancia de especies muestra que el
56\% de la comunidad presenta mayores valores de equidad (log normal
10\% y null 46\%).

La prueba I de Moran sugiere que las mirtáceas de esta localización
presentan patrones aglomerados al menos con la vecindad de primer orden,
con excepción de \emph{M. gatunensis} y \emph{E. galalonensis} que
muestran un patrón espacial aleatorio. Cabe destacar que para \emph{C.
schipii} existe una autocorrelación espacial positiva también para los
vecinos de segundo orden y en términos negativos del cuarto al sexto
orden; para \emph{E. nesiotica} y \emph{E. oerstediana} una
autocorrelacion negativa con vecinos de tercer a cuarto orden y de
cuarto a quinto orden, respectivamente, es decir, su abundancia
disminuye en esas vecindades cuando aumenta en la de primer orden y
viceversa.

La autocorrelación mediante la prueba de Mantel muestra que hay una
correlación espacial inducida por alguna variable en términos positivos
para el primer orden y en términos negativos para el tercer y sexto
orden (hasta 500 metros). La prueba Yo de Moran evidencia que \emph{C.
schipii} muestra un posible patrón de correlación inversa con el
\emph{B, Ca, Zn, N y pH}; para \emph{E. coloradoensis} infiere un patrón
en términos positivos con \emph{Ca} y \emph{N. min.} y de igual manera
para \emph{E. galalonensis} con la geomorfología de vaguada pct y
pendiente media.

\section{Discusión}\label{discusiuxf3n}

La comunidad de mirtáceas de la parcela permante de BCI posee una
riqueza de 7 especies con una abundancia de 5579 individuos, la mayor
abundancia se registró para las especies \emph{E. galalonensis}
(35.40\%) y \emph{E. oerstediana} (32.94\%) y la menor abundancia para
las especies \emph{P. friedrichsthalianum} (1.04\%) y \emph{M.
gatunensis} (1.004\%). En cada quadrat de la parcela BCI podemos
encontrar un mínimo de 58 y un máximo de 399 mirtáceas para un promedio
de 112 individuos por quadrat.

El 57\% de la riqueza, las especies del género \emph{Eugenia},
presentaron altos grados de asociación entre ellas, por lo que supone un
patrón de dependencia, a diferencia de las especies \emph{P.
friedriechsthalianum}, \emph{M. gatunensis} y \emph{C. schipii}
mostraron un posible patrón independiente, por lo que supone que se
presentan aleatoriamente en la muestra sin asociarse a las otras
especies.

Las mirtáceas de esta muestra, según el método Ward, se dividen en 4
grupos, cada uno con 2, 13, 15 y 20 sitios respectivamente; el grupo con
2 sitios es respaldado por los métodos de agrupamiento aglomerativo y el
\emph{bootstrap} multiescalar (BP de 76\% y un AU de 99\%) esto infiere
que es un grupo natural y real dentro de la localidad. Este agrupamiento
no se relaciona específicamente con alguna variable sino por la
preferencia de un conjunto de éstas, como lo son \emph{Al} y \emph{Fe}.

Según el análisis de especies indicadoras (IndVal), las especies
asociadas como diagnósticas para el agrupamiento Ward, fueron \emph{C.
schipii} para el grupo 3, \emph{E. coloradoensis} para el conjunto 1+2 y
\emph{E. oerstediana} para el grupo 3+4, esto puede ser debido a sus
altas abundancias presentes en estos grupos.

Los estimadores de riqueza demostraron que la completitud de muestra
para las mirtáceas en estudio fue alcanzda 100\%, lo mismo para los
grupos Ward, por lo que inferimos que el esfuerzo de muestreo surtió las
necesidas de lugar y no se espera encontrar más especies de en de esta
familia en BCI.

Los equidad de Pielou, los números de Hill destacan que la diversidad de
mirtáceas posee una correlación positiva con \emph{Al, P, Ca, Fe} y la
geomorfología de pendiente media.

Las especies que hacen contribución a la diversidad beta son \emph{C.
schipii} y \emph{E. oerstediana}, y los sitios que hacen contribución a
esta diversidad son el grupo 14 y 19, coincidencialmente un grupo Ward
que posee altas abundancias de las especies antes mencionadas, por lo
que este grupo guarda una estrecha relación con la presencia de estas
especies.

Las mirtáceas de esta localidad, según el I de Moran, presentan patrones
aglomerados para vecinos de primer orden que implican hasta 50 sitios,
con excepción de \emph{M. gatunencis}, que como había mencionado antes,
presentó un patrón aleatorio; en el caso de \emph{C. schipii}, presenta
patrón más aglomerado que las demás debido a que presenta una
correlación positiva también con los vecinos de segundo orden y en
términos negativos con los vecinos del cuarto al sexto orden, es decir
que aumenta o disminuye de forma inversa en estos lugares en relación
con el primer y segundo orden.

El modelo de abundancia de especies muestra que el 56\% de la comunidad
presenta mayores valores de equidad (log normal 10\% y null 46\%),
parece una cifra razonable debido a que el 48\% de los quadrats poseen
valores 7 y 6 de riqueza, lo que infiere que los modelos de distribución
de especies (SDM) parecen estar prediciendo bien la ocurrencia de dichas
especies.

\section{Agradecimientos}\label{agradecimientos}

\section{Información de soporte}\label{informaciuxf3n-de-soporte}

\ldots

\section{\texorpdfstring{\emph{Script}
reproducible}{Script reproducible}}\label{script-reproducible}

\ldots

\section*{Referencias}\label{referencias}
\addcontentsline{toc}{section}{Referencias}

\hypertarget{refs}{}
\hypertarget{ref-mapview}{}
Appelhans, T., Detsch, F., Reudenbach, C., \& Woellauer, S. (2019).
\emph{Mapview: Interactive viewing of spatial data in r}. Retrieved from
\url{https://CRAN.R-project.org/package=mapview}

\hypertarget{ref-jose_ramon_martinez_batlle_2020_4402362}{}
Batlle, J. R. M. (2020). biogeografia-master/scripts-de-analisis-BCI:
Long coding sessions (Version v0.0.0.9000).
\url{https://doi.org/10.5281/zenodo.4402362}

\hypertarget{ref-borcard2018numerical}{}
Borcard, F. ~. ois y L., Daniel y Gillet. (2018). \emph{Ecología
numérica con r}. Springer.

\hypertarget{ref-leaflet}{}
Cheng, J., Karambelkar, B., \& Xie, Y. (2018). \emph{Leaflet: Create
interactive web maps with the javascript 'leaflet' library}. Retrieved
from \url{https://CRAN.R-project.org/package=leaflet}

\hypertarget{ref-webcenso}{}
Hubbell, S., Condit, R., \& Foster, R. (2021). Forest Census Plot on
Barro Colorado Island. Retrieved May 5, 2021, from
\url{http://ctfs.si.edu/webatlas/datasets/bci/}

\hypertarget{ref-vegan}{}
Oksanen, J., Blanchet, F. G., Friendly, M., Kindt, R., Legendre, P.,
McGlinn, D., \ldots{} Wagner, H. (2019). \emph{Vegan: Community ecology
package}. Retrieved from \url{https://CRAN.R-project.org/package=vegan}

\hypertarget{ref-sf}{}
Pebesma, E. (2018). Simple Features for R: Standardized Support for
Spatial Vector Data. \emph{The R Journal}, \emph{10}(1), 439--446.
\url{https://doi.org/10.32614/RJ-2018-009}

\hypertarget{ref-perez2005metodologia}{}
Pérez, R., Aguilar, S., Condit, R., Foster, R., Hubbell, S., \& Lao, S.
(2005). Metodologia empleada en los censos de la parcela de 50 hectareas
de la isla de barro colorado, panamá. \emph{Centro de Ciencias
Forestales Del Tropico (CTFS) Y Instituto Smithsonian de Investigaciones
Tropicales (STRI)}, 1--24.

\hypertarget{ref-citadeR}{}
R Core Team. (2019). \emph{R: A language and environment for statistical
computing}. Retrieved from \url{https://www.R-project.org/}

\hypertarget{ref-restrepo2007pearson}{}
Restrepo, L. F., \& González, J. (2007). From pearson to spearman.
\emph{Revista Colombiana de Ciencias Pecuarias}, \emph{20}(2), 183--192.

\hypertarget{ref-bci_video}{}
Smithsonian Tropical Research Institute. (2010). \emph{Un vistazo a la
ciencia y los cientificos que trabajan en la isla de Barro Colorado}.
urlhttps://youtu.be/bN54RGtxFeM.

\hypertarget{ref-tidyverse}{}
Wickham, H. (2017). \emph{Tidyverse: Easily install and load the
'tidyverse'}. Retrieved from
\url{https://CRAN.R-project.org/package=tidyverse}

\hypertarget{ref-wilson2010myrtaceae}{}
Wilson, P. G. (2010). Myrtaceae. In \emph{Flowering plants. eudicots}
(pp. 212--271). Springer.




\newpage
\singlespacing 
\end{document}
